% -*- coding: utf-8 -*-
\documentclass[11pt]{article}
\usepackage[utf8]{inputenc}
\usepackage[T1]{fontenc}
\usepackage{fixltx2e}
\usepackage{graphicx}
\usepackage{longtable}
\usepackage{float}
\usepackage{wrapfig}
\usepackage{soul}
\usepackage{t1enc}
\usepackage{textcomp}
\usepackage{marvosym}
\usepackage{wasysym}
\usepackage{latexsym}
\usepackage{amssymb}
\usepackage{hyperref}
\tolerance=1000 
\usepackage{color}
\usepackage{listings}
\usepackage[bottom=2cm]{geometry}
\hypersetup{colorlinks,pdffitwindow,linkcolor=blue}
\definecolor{lbcolor}{rgb}{0.95,0.95,0.95}\definecolor{cblue}{rgb}{0.,0.0,0.6}\definecolor{lblue}{rgb}{0.1,0.1,0.4}
\lstset{language=C++,showspaces=false,showstringspaces=false,captionpos=t,literate={>>}{\ensuremath{>>}}1,mathescape}
\lstset{basicstyle=\small\ttfamily}
\lstset{lineskip=-2pt}
\lstset{keywordstyle=\color{red}\bfseries}
\lstset{emph={inline},emphstyle=\color{red}\bfseries}
\lstset{commentstyle=\ttfamily\color{cblue}}
\lstset{backgroundcolor=\color{lbcolor},rulecolor=}
\lstset{frame=single,framerule=0.5pt}
\lstset{belowskip=\smallskipamount}
\lstset{aboveskip=\smallskipamount}
\providecommand{\alert}[1]{\textbf{#1}}

\title{Report on Lab 2 of SDTM \\
 By\\
  
  \textbf{}}
\author{AMAKOR Augustina Chidinma and KARIMI Soufiane}
\date{October 19th}


\begin{document}

\maketitle
\setcounter{tocdepth}{2}
\tableofcontents
\vspace*{1cm}

\setcounter{tocdepth}{2}
\vspace*{1cm}




\section{Create Project}

We started our project by creating a directory in the \texttt{HOME} Directory named \texttt{Lab2}, and we also created a \texttt{CMakelists.txt} as shown below:\\

\begin{lstlisting}[language={sh}]
(base)karimi@MacBook-Pro-de-Karimi% mkdir Lab2
(base)karimi@MacBook-Pro-de-Karimi% cd Lab2
(base)karimi@MacBook-Pro-de-Karimi% cat > CMakeLists.txt
\end{lstlisting}
In the \texttt{CMakelists.txt}, we included the below codes:
\begin{lstlisting}[language={sh}]
cmake_minimum_required(VERSION 2.8.11)
set(CMAKE_BUILD_TYPE Release)
project(Lab2)
\end{lstlisting}

The file at the root makes sure that the version of \texttt{cmake} is currently recent(2.8.11), defines the name of the project and indicates to compile in release mode.
Then we compiled by \texttt{cmake .} then \texttt{make}

\section{Installation of Eigen}
We untared the \texttt{eigen-3.4.0.tar} and add it to our project directory \texttt{Lab2}. Then we included it into the \texttt{CMakelists.txt} as an external one as shown below:\\

\begin{lstlisting}[language={sh}]
include(ExternalProject)
ExternalProject_Add(
Eigen
SOURCE_DIR SOURCE_DIR $\textrm{\$}${Lab2_SOURCE_DIR}/eigen-3.4.0
INSTALL_COMMAND echo "Skipping install")
\end{lstlisting}

We compiled the above using \texttt{make} and the eigen library was installed successfully.


\section{Eigen Library}
We then add the eigen directory using the code below to the \texttt{CMakelists.txt}:

\begin{lstlisting}[language={sh}]
INCLUDE_DIRECTORIES($\textrm{\$}${Lab2_SOURCE_DIR}/eigen-3.4.0)
\end{lstlisting}

\section{Quick Test}
We created a subdirectory \texttt{tests} where we added some tests \texttt{test1.cpp} and \texttt{test2.cpp}  to check Eigen Library. For this, we updated the main \texttt{CMakelists.txt} in the \texttt{Lab2} directory by adding;
\begin{lstlisting}[language={sh}]
add_subdirectory(tests)
\end{lstlisting}
\vspace{5mm}
Also, we create a \texttt{CMakelists.txt} in the subdirectory \texttt{tests}  as shown below:
\begin{lstlisting}[language={sh}]
add_executable(test1 test1.cpp)
add_executable(test2 test2.cpp)
enable_testing()
add_test(NAME Test1 COMMAND test1)
add_test(NAME Test2 COMMAND test2)
\end{lstlisting}
Then we compiled our tests with \texttt{make} respectively as shown below:\\
\texttt{test1.cpp} 
\begin{lstlisting}[language=c++]
#include <iostream>
#include <Eigen/Dense>
 
using Eigen::MatrixXd;
 
int main()
{
  MatrixXd m(2,2);
  m(0,0) = 3;
  m(1,0) = 2.5;
  m(0,1) = -1;
  m(1,1) = m(1,0) + m(0,1);
  std::cout << m*m << std::endl;
}
\end{lstlisting}
Output:\texttt{./test1}
\begin{lstlisting}[language={sh}]
 6.5  -4.5
11.25 -0.25
\end{lstlisting}
\texttt{test2.cpp} 
\begin{lstlisting}[language=c++]
#include <iostream>
#include <Eigen/Dense>
 
using Eigen::MatrixXd;
 
int main()
{
  VectorXd v(2);
  v(0) = 4;
  v(1) = v(0) - 1;
  std::cout << v << std::endl;
}
\end{lstlisting}
Output:\texttt{./test2}
\begin{lstlisting}[language={sh}]
4
3
\end{lstlisting}
Our Eigen Library was successfully imported since we had no error.\\

\section{Testing Performance of Eigen with Time}
Here we created new tests of Matrix by Matrix multiplication of size(10,100,1000) and Matrix by vector multiplication of size(10,100,1000) to test the performance of the eigen. Also, we created a benchmark \texttt{header.cpp} that contains the \texttt{include directives} for easy inclusion into our tests as shown below;\\
\vspace{5mm}
\texttt{./tests/header.cpp} 
\begin{lstlisting}[language=c++]
#include <iostream>
#include <Eigen/Dense>
# include <boost/timer/timer.hpp>
using namespace Eigen ;
using namespace boost :: timer ;
\end{lstlisting}
\vspace{5mm}
\texttt{./tests/test\char`_timer1.cpp}\\
Initialise a matrix $A$ of size($n$,$n$) where $n$ $\in$ ($10$,$100$,$1000$), do the multiplication $A*A$, and return the time compilation for each size.
\begin{lstlisting}[language=c++]
#include header.cpp
MatrixXd multiplication_matrix(int size){
  MatrixXd A(size,size);
  for (int row = 0; row <size; ++row){
   for (int col = 0; col <size; ++col)
    {
     A(row,col) = row+col;
    }
  }
  return A*A;
}
int main()
{
  cpu_timer timer;
  std::cout << "Here is A*A:\n" << multiplication_matrix(10)
  <<std::endl;
  cpu_times times10 = timer.elapsed();
  timer.start();
  std::cout << "Here is A*A:\n" << multiplication_matrix(100)
  << std::endl;
  cpu_times times100 = timer.elapsed();
  timer.start();  
  std::cout << "Here is A*A:\n" << multiplication_matrix(1000) 
  << std::endl;
  cpu_times times1000 = timer.elapsed();
  std::cout <<"wall time by size:10,100,1000 : "<<times10.wall<<","
  <<times100.wall<<","<<times1000.wall<<'\n';
  std::cout <<"system time by size:10,100,1000 : "<<times10.system
  <<","<<times100.system<<","<<times1000.system<<'\n';
  std::cout <<"user time by size:10,100,1000 : "<<times10.user
  <<","<<times100.user<<","<<times1000.user<<'\n'; 
}
\end{lstlisting}
\vspace{5mm}
\texttt{./tests/test\char`_timer2.cpp} \\
Initialise a matrix $A$ of size($n$,$n$) and a vector $V$ of size $n$ where $n$ $\in$ ($10$,$100$,$1000$), do the multiplication $A*v$, and return the time compilation for each size.
\begin{lstlisting}[language=c++]
#include header.cpp
VectorXd multiplication_matrix_vector(int size){
  MatrixXd A(size,size);
  for (int row = 0; row <size; ++row){
   for (int col = 0; col <size; ++col)
    {
     A(row,col) = row+col;
    }
  }
  VectorXd v(size);
  for (int row = 0; row <size; ++row){
       v(row) = row;
  }
  return A*v;
}
int main()
{
  cpu_timer timer;
  std::cout << "Here is A*v:\n" << multiplication_matrix_vector(10)
  << std::endl;
  cpu_times times10 = timer.elapsed();
  timer.start();
  std::cout << "Here is A*v:\n" << multiplication_matrix_vector(100)
  << std::endl;
  cpu_times times100 = timer.elapsed();
  timer.start();  
  std::cout << "Here is A*v:\n" << multiplication_matrix_vector(1000)
  << std::endl;\vspace{5mm}
  cpu_times times1000 = timer.elapsed();
  std::cout <<"wall time by size:10,100,1000 : "<<times10.wall<<","
  <<times100.wall<<","<<times1000.wall<<'\n';
  std::cout <<"system time by size:10,100,1000 : "<<times10.system
  <<","<<times100.system<<","<<times1000.system<<'\n';
  std::cout <<"user time by size:10,100,1000 : "<<times10.user
  <<","<<times100.user<<","<<times1000.user<<'\n'; 
}
\end{lstlisting}
\vspace{5mm}
Here, we tested the performance (time complexity)  of all our eigen tests using (\texttt{Boost.Timer}). To do this, we included all the \texttt{system and timer} components of the Boost for the individual tests into the \texttt{CMakelists.txt} of the  \texttt{tests} directory as shown below: 

\begin{lstlisting}[language={sh}]
set(Boost_USE_STATIC_LIBS ON)
find_package(Boost COMPONENTS system timer unit_test_framework REQUIRED)
add_executable(test_timer1 test_timer1.cpp)
add_executable(test_timer2 test_timer2.cpp)

target_link_libraries(test_timer1
        $\textrm{\$}${Boost_LIBRARIES}
        $\textrm{\$}${Boost_UNIT_TEST_FRAMEWORK_LIBRARY}
        $\textrm{\$}${Boost_TIMER_LIBRARY}
        $\textrm{\$}${Boost_SYSTEM_LIBRARY})
target_link_libraries(test_timer2
        $\textrm{\$}${Boost_LIBRARIES}
        $\textrm{\$}${Boost_UNIT_TEST_FRAMEWORK_LIBRARY}
        $\textrm{\$}${Boost_TIMER_LIBRARY}
        $\textrm{\$}${Boost_SYSTEM_LIBRARY})
enable_testing()
add_test(test_timer1 test_timer1)
add_test(test_timer2 test_timer2)
\end{lstlisting}

We compile then our tests by \texttt{make} and we get the outputs shown below:\\
\texttt{./test\char`_timer1}
\begin{lstlisting}[language={sh}]
wall time by size:10,100,1000 : 308039,18446740,1058655227
system time by size:10,100,1000 : 0,0,50000000
user time by size:10,100,1000 : 0,10000000,920000000
\end{lstlisting}
\texttt{./test\char`_timer2}
\begin{lstlisting}[language={sh}]
wall time by size:10,100,1000 : 184896,452254,11537544
system time by size:10,100,1000 : 0,0,0
user time by size:10,100,1000 : 0,0,10000000
\end{lstlisting}

We remark that the time compilation increases in size.

\section{Optimization of codes using flags}
We compared our previous time of code compilation by updating the \texttt{CMakelists.txt} in the main directory with \begin{lstlisting}[language={sh}]
set(CMAKE_CXX_FLAGS "-g")
\end{lstlisting} using test1 results with the following flags\\
\texttt{-g}
\begin{lstlisting}[language={sh}]
wall time by size:10,100,1000 : 491531,21054504,1033193165
system time by size:10,100,1000 : 0,0,50000000
user time by size:10,100,1000 : 0,10000000,910000000
\end{lstlisting}
\vspace{5mm}
\texttt{-O2}
\begin{lstlisting}[language={sh}]
wall time by size:10,100,1000 : 555540,19407981,1034552431
system time by size:10,100,1000 : 0,0,50000000
user time by size:10,100,1000 : 0,10000000,920000000
\end{lstlisting}
\vspace{5mm}
\texttt{-O3}
\begin{lstlisting}[language={sh}]
wall time by size:10,100,1000 : 201318,13564393,1033549776
system time by size:10,100,1000 : 0,0,50000000
user time by size:10,100,1000 : 0,10000000,910000000
\end{lstlisting}
\vspace{5mm}
\texttt{-O2 -msse2}
\begin{lstlisting}[language={sh}]
wall time by size:10,100,1000 : 194555,12009848,1026738239
system time by size:10,100,1000 : 0,0,50000000
user time by size:10,100,1000 : 0,10000000,900000000
\end{lstlisting}
\vspace{5mm}
\texttt{-O3 -msse2}
\begin{lstlisting}[language={sh}]
wall time by size:10,100,1000 : 213191,12832374,1030161815
system time by size:10,100,1000 : 0,0,50000000
user time by size:10,100,1000 : 0,10000000,900000000
\end{lstlisting}
Using the flags, we observed that \texttt{-O2 -msse2} and \texttt{-O3 -msse2} are equivalent, so is also \texttt{-O2} and \texttt{-O3}. \texttt{-g} performs better than the two former but the best to optimize for code size and execution time is \texttt{-O2 -msse2}.

\section{Comparing Eigen and Boost.ublas by Time Compilation}
Using this time \texttt{Boost.ublas}, we created new tests of Matrix by Matrix multiplication of size(10,100,1000) and Matrix by vector multiplication of size(10,100,1000) to test the performance of the ublas.\\
Also, we created a benchmark \texttt{header2.cpp} that contains the \texttt{include directives} for easy inclusion into our tests as shown below;\\
\texttt{./tests/header2.cpp}
\begin{lstlisting}[language=c++]
#include <boost/timer/timer.hpp>
#include <boost/numeric/ublas/matrix.hpp>
#include <boost/numeric/ublas/io.hpp>
using namespace boost::timer;
using namespace boost::numeric::ublas;
\end{lstlisting}
\vspace{5mm}
\texttt{./tests/test\char`_ublas1.cpp}\\
Initialise a matrix $A$ of size($n$,$n$) where $n$ $\in$ ($10$,$100$,$1000$), do the multiplication $A*A$, and return the time compilation for each\vspace{5mm} size.\\
\begin{lstlisting}[language=c++]
#include header2.cpp
matrix<double> multiplication_matrix(int size){
  matrix<double> A(size,size);
  for (int row = 0; row <size; ++row){
   for (int col = 0; col <size; ++col)
    {
     A(row,col) = row+col;
    }
  }
  return prod(A,A);
}
int main()
{
  cpu_timer timer;
  std::cout << "Here is A*A:\n" << multiplication_matrix(10)
  <<std::endl;
  cpu_times times10 = timer.elapsed();
  timer.start();
  std::cout << "Here is A*A:\n" << multiplication_matrix(100)
  << std::endl;
  cpu_times times100 = timer.elapsed();
  timer.start();  
  std::cout << "Here is A*A:\n" << multiplication_matrix(1000) 
  << std::endl;
  cpu_times times1000 = timer.elapsed();
  std::cout <<"wall time by size:10,100,1000 : "<<times10.wall<<","
  <<times100.wall<<","<<times1000.wall<<'\n';
  std::cout <<"system time by size:10,100,1000 : "<<times10.system
  <<","<<times100.system<<","<<times1000.system<<'\n';
  std::cout <<"user time by size:10,100,1000 : "<<times10.user
  <<","<<times100.user<<","<<times1000.user<<'\n'; 
}
\end{lstlisting}
\vspace{5mm}
\texttt{./tests/test\char`_ublas2.cpp} \\
Initialise a matrix $A$ of size($n$,$n$) and a vector $V$ of size $n$ where $n$ $\in$ ($10$,$100$,$1000$), do the multiplication $A*v$, and return the time compilation for each size.\\
\begin{lstlisting}[language=c++]
#include header2.cpp
vector<double> multiplication_matrix_vector(int size){
  matrix<double> A(size,size);
  for (int row = 0; row <size; ++row){
   for (int col = 0; col <size; ++col)
    {
     A(row,col) = row+col;
    }
  }
  vector<double> v(size);
  for (int row = 0; row <size; ++row){
       v(row) = row;
  }
  return prod(A,v);
}
int main()
{
  cpu_timer timer;
  std::cout << "Here is A*v:\n" << multiplication_matrix_vector(10)
  << std::endl;
  cpu_times times10 = timer.elapsed();
  timer.start();
  std::cout << "Here is A*v:\n" << multiplication_matrix_vector(100)
  << std::endl;
  cpu_times times100 = timer.elapsed();
  timer.start();  
  std::cout << "Here is A*v:\n" << multiplication_matrix_vector(1000)
  << std::endl;
  cpu_times times1000 = timer.elapsed();
  std::cout <<"wall time by size:10,100,1000 : "<<times10.wall<<","
  <<times100.wall<<","<<times1000.wall<<'\n';
  std::cout <<"system time by size:10,100,1000 : "<<times10.system
  <<","<<times100.system<<","<<times1000.system<<'\n';
  std::cout <<"user time by size:10,100,1000 : "<<times10.user
  <<","<<times100.user<<","<<times1000.user<<'\n'; 
}
\end{lstlisting}

\vspace{9mm}
Here, we tested the performance (time complexity)  of all our ublas tests using (\texttt{Boost}). To do this, we included all the \texttt{system and timer and Boost Libraries} for the individual tests into the \texttt{CMakelists.txt} of the  \texttt{tests} directory as shown below: \\

\begin{lstlisting}[language={sh}]
set(Boost_USE_STATIC_LIBS ON)
find_package(Boost COMPONENTS system timer unit_test_framework REQUIRED)
add_executable(test_timer1 test_timer1.cpp)
add_executable(test_timer2 test_timer2.cpp)
add_executable(test_ublas1 test_ublas1.cpp)
add_executable(test_ublas2 test_ublas2.cpp)
target_link_libraries(test_timer1
        $\textrm{\$}${Boost_LIBRARIES}
        $\textrm{\$}${Boost_UNIT_TEST_FRAMEWORK_LIBRARY}
        $\textrm{\$}${Boost_TIMER_LIBRARY}
        $\textrm{\$}${Boost_SYSTEM_LIBRARY})
target_link_libraries(test_timer2
        $\textrm{\$}${Boost_LIBRARIES}
        $\textrm{\$}${Boost_UNIT_TEST_FRAMEWORK_LIBRARY}
        $\textrm{\$}${Boost_TIMER_LIBRARY}
        $\textrm{\$}${Boost_SYSTEM_LIBRARY})
target_link_libraries(test_ublas1
        $\textrm{\$}${Boost_LIBRARIES}
        $\textrm{\$}${Boost_TIMER_LIBRARY}
        $\textrm{\$}${Boost_FILESYSTEM_LIBRARY}
        $\textrm{\$}${Boost_SYSTEM_LIBRARY}
        $\textrm{\$}${Boost_UNIT_TEST_FRAMEWORK_LIBRARY}
                       )
target_link_libraries(test_ublas2
        $\textrm{\$}${Boost_LIBRARIES}
        $\textrm{\$}${Boost_TIMER_LIBRARY}
        $\textrm{\$}${Boost_FILESYSTEM_LIBRARY}
        $\textrm{\$}${Boost_SYSTEM_LIBRARY}
        $\textrm{\$}${Boost_UNIT_TEST_FRAMEWORK_LIBRARY}
                       )
enable_testing()
add_test(test_timer1 test_timer1)
add_test(test_timer2 test_timer2)
add_test(test_ublas1 test_ublas1)
add_test(test_ublas2 test_ublas2)
\end{lstlisting}
\vspace{5mm}
We compiled our tests by \texttt{make} and we got the outputs shown below:\\
\texttt{./test\char`_ublas1}
\begin{lstlisting}[language={sh}]
wall time by size:10,100,1000 : 199058,14802150,1897406463
system time by size:10,100,1000 : 0,0,60000000
user time by size:10,100,1000 : 0,0,1580000000
\end{lstlisting}
\texttt{./test\char`_ublas2}
\begin{lstlisting}[language={sh}]
wall time by size:10,100,1000 : 176577,177500,11335523
system time by size:10,100,1000 : 0,0,0
user time by size:10,100,1000 : 0,0,0
\end{lstlisting}
we remark that for large size matrix(1000), the compilation using Eigen is faster, which is not true for small sizes(10,100).   
\section{Processors manipulation}

We execute our tests using 1,2,3,4 processors with the commands below:
\begin{lstlisting}[language={sh}]
taskset 0X1 make
taskset 0X2 make
taskset 0X3 make
taskset 0X4 make
\end{lstlisting}
 and we observe the following outputs for the test \texttt{test\char`_timer1}:
\texttt{Using one processor}
\begin{lstlisting}[language={sh}]
wall time by size:10,100,1000 : 217583,14074826,2193977452
system time by size:10,100,1000 : 0,0,30000000
user time by size:10,100,1000 : 0,10000000,1530000000
\end{lstlisting}
\texttt{Using two processors}
\begin{lstlisting}[language={sh}]
wall time by size:10,100,1000 : 150798,14689499,2131581800
system time by size:10,100,1000 : 0,0,40000000
user time by size:10,100,1000 : 0,10000000,1500000000
\end{lstlisting}
\texttt{Using three processors}
\begin{lstlisting}[language={sh}]
wall time by size:10,100,1000 : 677481,26196988,2122697280
system time by size:10,100,1000 : 0,0,70000000
user time by size:10,100,1000 : 0,10000000,1460000000
\end{lstlisting}
\texttt{Using four processors}
\begin{lstlisting}[language={sh}]
wall time by size:10,100,1000 : 391060,26761858,2124255717
system time by size:10,100,1000 : 0,0,90000000
user time by size:10,100,1000 : 0,10000000,1450000000
\end{lstlisting}

We remark that the time of compilation decreased when augmenting the number of working processors.

\end{document}
%%% Local Variables:
%%% mode: latex
%%% TeX-PDF-mode: t
%%% TeX-parse-self: t
%%% x-symbol-8bits: nil
%%% TeX-auto-regexp-list: TeX-auto-full-regexp-list
%%% TeX-master: "tp1.tex"
%%% ispell-local-dictionary: "american"
%%% End:













